\documentclass{article}
\usepackage{multirow}
\usepackage{geometry}
 \geometry{
 a4paper,
 total={170mm,257mm},
 left=25mm,
 right=25mm,
 top=20mm,
 bottom=30mm
 }
\begin{document}
\title{Evaluation of benchmarks}
%\institute{PDC Center for High-Performance computing, KTH, Stockholm}

\maketitle

\section*{Throughput and strong-scaling benchmarks}

All application benchmarks belong to one of two classes: \textit{throughput} and \textit{strong-scaling}.

\subsubsection*{Throughput}

\begin{itemize}
    \item Use case: regular HPC research where a certain number of jobs need to be completed in order to finish a research project, without wasting core hours. 
    \item Figure of merit (result): Number of jobs that run on the system in one day.
    \item Formula:
    \begin{equation}
        n_{\textrm{jobs}}(b) = \frac{t_{\textrm{day}}}{t(b)} \cdot \frac{N_{\textrm{tot}}}{N(b)}       
    \end{equation}
    where:
    \begin{itemize}
        \item $t_{\textrm{day}}$ = time in a day (86400 s)
        \item $t(\textrm{b})$ = average time in seconds to run one job for benchmark $b$
        \item $N_{\textrm{tot}}$ = total number of nodes tendered
        \item $N(b)$ = number of nodes used to run benchmark $b$
        \item $n_{\textrm{jobs}}(b)$ = number of jobs that can be run in a day for benchmark $b$. This result must be reported for each throughput benchmark in the benchmark matrix spreadsheet.
    \end{itemize}
    \item All throughput benchmarks have a \textbf{minimum performance} to ensure that more than one node needs to be used.

\end{itemize}

The throughput tests represent a trade-off between maximizing the number of nodes in the system, minimizing the time it takes to run one job and minimize the number of nodes used for each job. Each
throughput benchmark case has a specified minimum performance which most likely requires it to be run on more than one node.

\subsubsection*{Strong Scaling}

\begin{itemize}
    \item Use case: Researcher needs to finish a massively parallel job in the shortest possible time. The researcher has access to the entire system.
    \item Figure of merit (result): Number of jobs (each job solves the problem in the shortest possible time) that run on the system in one day.
    \item Formula:
    \begin{equation}
        n_{\textrm{jobs}}(b) = \frac{t_{\textrm{day}}}{t(b)} 
    \end{equation}
    where:
    \begin{itemize}
        \item $t_{\textrm{day}}$ = time in a day (86400 s)
        \item $t(b)$ = average time in seconds to run one job in the shortest possible time for benchmark $b$
        \item $n_{\textrm{jobs}}(b)$ = number of jobs that can be run in one day for benchmark $b$. This result must be reported for each strong-scaling benchmark in the benchmark matrix spreadsheet.
    \end{itemize}
\end{itemize}

In the strong scaling benchmarks the goal is simply to minimize the runtime of the benchmark cases by using as many nodes/cores/devices as required to maximize performance.\\

The calculation of the throughput and strong-scaling figures of merit
is clarified in the documentation for each benchmark case.


\end{document}
